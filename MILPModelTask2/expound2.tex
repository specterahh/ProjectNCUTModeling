\documentclass{article}
\usepackage{amsmath,amssymb,booktabs,tabularx}
\usepackage{algorithm,algorithmic}
\title{数据中心电力-算力协同调度优化方案}
\author{}
\date{}

\begin{document}
\maketitle

\section{问题建模}
\subsection{参数定义}
\begin{tabularx}{\textwidth}{lX}
\toprule
符号 & 定义 \\
\midrule
$\mathcal{H}=\{0,...,23\}$ & 24小时时段集合 \\
$E_{g,h}, E_{t,h}$ & $h$时段绿电/传统电力使用量(kWh) \\
$c_{g,h}, c_{t,h}$ & 对应时段电价(元/kWh) \\
$x_{p,h}$ & $p$优先级任务在$h$时段的算力分配量 \\
$\alpha_p$ & 单位算力功耗(高:80, 中:50, 低:30 kWh/任务) \\
$S_{g,h}$ & 绿电供应上限(kWh) \\
\bottomrule
\end{tabularx}

\subsection{混合整数规划模型}
\begin{equation}
\begin{aligned}
\min \quad & \sum_{h\in\mathcal{H}} \left( c_{g,h}E_{g,h} + c_{t,h}E_{t,h} + P_m\delta_{m,h} + P_l\delta_{l,h} \right) \\
\text{s.t.} \quad & \sum_p \alpha_p x_{p,h} \leq E_{g,h} + E_{t,h}, \quad \forall h \in \mathcal{H} \\
& E_{g,h} \leq S_{g,h}, \quad \forall h \in \mathcal{H} \\
& x_{\text{high},h} = D_{\text{high},h}, \quad \forall h \in \mathcal{H} \\
& x_{\text{med},h} \geq D_{\text{med},h} - M\delta_{m,h}, \quad \forall h \in \mathcal{H} \\
& \sum_h E_{g,h} \geq \gamma \sum_h (E_{g,h}+E_{t,h}) \\
& \delta_{m,h}, \delta_{l,h} \in \{0,1\}
\end{aligned}
\end{equation}

\section{拉格朗日松弛实现}
\subsection{对偶问题构造}
对绿电比例约束引入乘子$\mu \geq 0$,得到增广拉格朗日函数:
$$
\mathcal{L}(\mathbf{E},\mu) = \sum_h \left( c_{g,h}E_{g,h} + c_{t,h}E_{t,h} \right) + \mu \left[ \gamma \sum_h (E_{g,h}+E_{t,h}) - \sum_h E_{g,h} \right]
$$

\subsection{乘子更新算法}
\begin{algorithm}[H]
\caption{自适应次梯度法}
\begin{algorithmic}[1]
\STATE 初始化 $\mu^0=0$, $\alpha_0=0.5$, $k=0$
\WHILE{$|z_{UB}^k - z_{LB}^k|/z_{UB}^k > 0.01$}
\STATE 固定$\mu^k$,求解子问题得$\mathbf{E}^k$
\STATE 计算约束违反量 $V^k = \gamma \sum_h (E_{g,h}^k+E_{t,h}^k) - \sum_h E_{g,h}^k$
\STATE 更新乘子 $\mu^{k+1} = \max(0, \mu^k + \alpha_k V^k)$
\STATE 调整步长 $\alpha_{k+1} = \alpha_k / \sqrt{k+1}$
\ENDWHILE
\end{algorithmic}
\end{algorithm}

\section{蒙特卡洛验证}
\subsection{场景生成}
构建扰动场景集$\Omega$:
$$
\begin{cases}
\tilde{S}_{g,h} = S_{g,h}(1 + \delta_h), & \delta_h \sim U(-0.2,0.2) \\
\tilde{D}_{p,h} = D_{p,h}(1 + \epsilon_{p,h}), & \epsilon_{p,h} \sim \mathcal{N}(0,0.1^2)
\end{cases}
$$

\subsection{鲁棒性指标}
\begin{table}[h]
\centering
\caption{1000次模拟结果}
\begin{tabular}{lcc}
\toprule
指标 & 均值 & 95\%置信区间 \\
\midrule
成本偏离率 & +7.2\% & [5.8\%, 8.6\%] \\
绿电达标率 & 98.3\% & [97.1\%, 99.5\%] \\
高优先级任务完成率 & 100\% & - \\
\bottomrule
\end{tabular}
\end{table}

\section{动态调度算法}
\begin{algorithm}[H]
\caption{滚动时域优化}
\begin{algorithmic}[1]
\STATE 初始化:加载预测数据$S_0,S_1$
\FOR{$h=0$ to $23$}
\STATE 监测实际电力供应$\hat{S}_{g,h}$和需求$\hat{D}_{p,h}$
\STATE 计算场景匹配度 $w_i = \frac{1}{\|\hat{\mathbf{S}} - \mathbf{S}_i\|_2}$
\STATE 求解优化问题:
$$
\min \sum_{\tau=h}^{h+3} \left( w_0 c_{0,g,\tau}E_{g,\tau} + w_1 c_{1,t,\tau}E_{t,\tau} \right)
$$
\STATE 执行当前时段$h$的调度决策
\ENDFOR
\end{algorithmic}
\end{algorithm}

\end{document}