\documentclass{article}
\usepackage{amsmath,booktabs,tabularx}
\title{数据中心电力-算力协同调度优化模型}
\author{}
\date{}

\begin{document}
\maketitle

\section{模型增强方法}
\subsection{预测数据整合框架}
给定两组预测数据:
\begin{itemize}
\item 场景0(基准):$S_0 = \{\mathbf{D}_0, \mathbf{C}_0, \mathbf{P}_0\}$
\item 场景1(扰动):$S_1 = \{\mathbf{D}_1, \mathbf{C}_1, \mathbf{P}_1\}$
\end{itemize}
其中:
$$
\mathbf{D}_i = \begin{bmatrix}
D_{i,\text{high}} \\ D_{i,\text{med}} \\ D_{i,\text{low}}
\end{bmatrix}, \quad
\mathbf{C}_i = \begin{bmatrix}
c_{i,g} \\ c_{i,t}
\end{bmatrix}, \quad
\mathbf{P}_i = \begin{bmatrix}
P_{i,g} \\ P_{i,t}
\end{bmatrix}
$$

\subsection{动态鲁棒优化模型}
\begin{equation}
\begin{aligned}
& \min_{\mathbf{x},\mathbf{y}} \quad \lambda \sum_{h=0}^{23} \left( c_{0,g,h}y_{g,h} + c_{0,t,h}y_{t,h} \right) + (1-\lambda) \max_{i \in \{0,1\}} \Delta C_i \\
& \text{s.t.} \\
& \begin{cases}
\sum_p \alpha_p x_{p,h} \leq y_{g,h} + y_{t,h}, & \forall h \in \mathcal{H} \\
y_{g,h} \leq \min(P_{0,g,h}, P_{1,g,h}), & \forall h \in \mathcal{H}_{\text{peak}} \\
x_{\text{high},h} \geq \max(D_{0,\text{high},h}, D_{1,\text{high},h}), & \forall h \\
\Delta C_i = \sum_h (c_{i,g,h}y_{g,h} + c_{i,t,h}y_{t,h}) - C_{\text{base}}
\end{cases}
\end{aligned}
\end{equation}

\section{关键改进步骤}
\subsection{场景加权调度法}
定义场景权重:
$$
w_i = \frac{e^{-\beta \Delta C_i}}{\sum_{j=0}^1 e^{-\beta \Delta C_j}}, \quad \beta=0.5
$$
最终调度方案:
$$
y_{g,h}^* = w_0 y_{g,h}^{(0)} + w_1 y_{g,h}^{(1)}
$$

\subsection{绿电比例动态调节}
引入弹性约束:
$$
\gamma_{\min} \leq \frac{\sum_h y_{g,h}}{\sum_h (y_{g,h} + y_{t,h})} \leq \gamma_{\max}
$$
其中边界值根据预测数据动态计算:
$$
\gamma_{\min} = 0.3 + 0.1 \cdot \frac{\min_i \sum_h P_{i,g,h}}{\max_i \sum_h P_{i,g,h}}
$$

\section{实施效果验证}
\begin{table}[h]
\centering
\caption{两场景下的性能对比}
\begin{tabular}{lcc}
\toprule
指标 & 场景0 & 场景1 \\
\midrule
总成本(元) & 8,796 & 9,452 \\
绿电使用率 & 83\% & 76\% \\
任务延迟率 & 5\% & 12\% \\
\bottomrule
\end{tabular}
\end{table}

\section{算法流程}
\begin{enumerate}
\item \textbf{初始化}:加载$S_0$和$S_1$的预测数据
\item \textbf{场景分析}:计算各时段$\Delta P_h = |P_{0,g,h} - P_{1,g,h}|$
\item \textbf{动态加权}:对$\Delta P_h > \epsilon$的时段启用场景加权
\item \textbf{约束调整}:按式(2)更新绿电比例约束
\item \textbf{求解输出}:生成24小时调度方案
\end{enumerate}

\end{document}