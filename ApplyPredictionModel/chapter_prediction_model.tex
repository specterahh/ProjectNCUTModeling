\documentclass[a4paper,12pt]{article}
\usepackage{amsmath}
\usepackage{amsfonts}
\usepackage{amssymb}
\usepackage{graphicx}
\usepackage{booktabs}
\usepackage{caption}
\usepackage{geometry}
\geometry{margin=1in}

% 数学相关包
\usepackage{mathtools}
\usepackage{bm}

% 表格相关包
\usepackage{array}
\usepackage{longtable}
\usepackage{colortbl}

% 设置语言支持
\usepackage[utf8]{inputenc}
\usepackage[T1]{fontenc}

% 字体配置(最后配置)
\usepackage{times} % 使用 Times 字体

\begin{document}

\section{Construction and Analysis of the Prediction Model}

\subsection{Selection and Theoretical Basis of the Prediction Model}

The prediction model is designed to capture both the periodic patterns and random fluctuations in the data. The observed data exhibits clear diurnal periodicity: new energy power generation peaks during the day (e.g., 3.2 MW at 11:00-12:00) and drops to zero at night (e.g., 00:00-04:00 and 21:00-24:00). Similarly, renewable energy prices are lower during the day (e.g., 0.3 RMB Yuan/kWh at 11:00-12:00) and higher at night (e.g., 0.6 RMB Yuan/kWh at 00:00-01:00). This periodicity suggests the use of a sinusoidal function to model diurnal trends. Additionally, real-world power and price data are subject to uncertainties (e.g., weather, policy, or demand variations), necessitating the inclusion of random perturbations.

A hybrid prediction model combining a sinusoidal periodic function and Gaussian noise is selected, leveraging time series analysis for periodicity and statistical theory for randomness. The model's advantages are:

\begin{itemize}
    \item The sinusoidal function effectively captures diurnal periodicity.
    \item Gaussian noise simulates random fluctuations, reflecting uncertainty.
    \item Constraints ensure the physical realism of predictions.
\end{itemize}

\subsection{Mathematical Formulation of the Prediction Model}

The prediction model is formulated for the predicted power supply \( \hat{P}(t) \), renewable energy price \( \hat{C}_r(t) \), and traditional energy price \( \hat{C}_t(t) \):

\begin{align}
\hat{P}(t) &= P(t) + A \cdot \sin\left(\frac{\pi (t - 6)}{12}\right) + \epsilon_p(t), \label{eq:power_pred} \\
\hat{C}_r(t) &= C_r(t) + B \cdot \sin\left(\frac{\pi (t - 6)}{12}\right) + \epsilon_r(t), \label{eq:renew_price_pred} \\
\hat{C}_t(t) &= C_t(t) + C \cdot \sin\left(\frac{\pi (t - 6)}{12}\right) + \epsilon_t(t), \label{eq:trad_price_pred}
\end{align}

where:
\begin{itemize}
    \item \( P(t) \), \( C_r(t) \), and \( C_t(t) \) are the original data for hour \( t \) (from 0 to 23), representing new energy power supply (MW), renewable energy price (RMB Yuan/kWh), and traditional energy price (RMB Yuan/kWh), respectively;
    \item \( A = 0.1 \) (amplitude for power supply fluctuation, in MW), \( B = 0.02 \) (amplitude for renewable energy price fluctuation, in RMB Yuan/kWh), and \( C = 0.05 \) (amplitude for traditional energy price fluctuation, in RMB Yuan/kWh);
    \item The sinusoidal term \( \sin\left(\frac{\pi (t - 6)}{12}\right) \) has a period of 12 hours, with a phase shift of 6 hours to align the peak with the observed data peak (around 11:00-12:00);
    \item \( \epsilon_p(t) \sim N(0, 0.1) \), \( \epsilon_r(t) \sim N(0, 0.02) \), and \( \epsilon_t(t) \sim N(0, 0.05) \) are Gaussian noise terms with means of 0 and standard deviations of 0.1, 0.02, and 0.05, respectively.
\end{itemize}

\subsection{Model Constraints and Key Time Period Handling}

To ensure physical realism, the following constraints are applied:

\begin{align}
\hat{P}(t) &\geq 0, \label{eq:power_constraint} \\
0.2 \leq \hat{C}_r(t) &\leq 0.8, \label{eq:renew_price_constraint} \\
0.4 \leq \hat{C}_t(t) &\leq 1.5. \label{eq:trad_price_constraint}
\end{align}

For the key time period at \( t = 11 \) (11:00-12:00), the predictions are fixed to match the observed data:

\begin{align}
\hat{P}(11) &= 3.2, \label{eq:power_key} \\
\hat{C}_r(11) &= 0.3, \label{eq:renew_price_key} \\
\hat{C}_t(11) &= 1.3. \label{eq:trad_price_key}
\end{align}

The predicted power in kilowatt-hours is calculated as:

\begin{equation}
\hat{P}_k(t) = 1000 \cdot \hat{P}(t). \label{eq:power_kwh}
\end{equation}

\subsection{Model Implementation and Results Generation}

The model is implemented using Python, with the Numpy library for generating sinusoidal fluctuations and Gaussian noise, and the Pandas library for data handling and CSV generation. The implementation steps are:

\begin{enumerate}
    \item Load the original 24-hour data from the CSV file.
    \item Compute predictions using equations \eqref{eq:power_pred}, \eqref{eq:renew_price_pred}, and \eqref{eq:trad_price_pred}, applying constraints \eqref{eq:power_constraint}, \eqref{eq:renew_price_constraint}, and \eqref{eq:trad_price_constraint}.
    \item Fix the predictions at 11:00-12:00 using equations \eqref{eq:power_key}, \eqref{eq:renew_price_key}, and \eqref{eq:trad_price_key}.
    \item Generate two independent sets of 24-hour predictions, saved as CSV files.
\end{enumerate}

Statistical analysis of the predictions shows an average power supply of 1.65 MW (range: 0.00 to 3.39 MW), an average renewable energy price of 0.46 RMB Yuan/kWh (range: 0.30 to 0.62 RMB Yuan/kWh), and an average traditional energy price of 0.92 RMB Yuan/kWh (range: 0.50 to 1.33 RMB Yuan/kWh). The key time period (11:00-12:00) matches the requirements exactly.

\subsection{Advantages and Limitations of the Model}

The prediction model offers the following advantages:

\begin{itemize}
    \item Accurate periodicity modeling through the sinusoidal function.
    \item Reasonable simulation of randomness via Gaussian noise.
    \item Scientifically grounded constraints ensuring physical realism.
\end{itemize}

However, the model has limitations:

\begin{itemize}
    \item Simplified periodicity assumption using a single 12-hour cycle.
    \item Gaussian noise assumption may not capture extreme events.
    \item Lack of multivariate interactions, such as task load influences.
\end{itemize}

\end{document}