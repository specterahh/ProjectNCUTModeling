\documentclass{article}
\usepackage{amsmath}
\usepackage{CJKutf8}

\begin{document}

\begin{CJK}{UTF8}{gbsn}

\section*{公式展示}

\subsection*{线性插值公式}
用于表1任务量数据补全的线性插值公式为:
\[
y(t) = y_1 + \frac{(y_2 - y_1)(t - t_1)}{t_2 - t_1}
\]
其中:
- \( y(t) \) 为待插值时间 \( t \) 处的任务量,
- \( y_1, y_2 \) 分别为相邻已知时段的任务量,
- \( t_1, t_2 \) 为对应时间点。

\subsection*{优化目标函数(参考)}
后续建模中拟采用的优化目标函数为:
\[
\min \sum_{t=0}^{23} (P_{g,t} \cdot C_{g,t} + P_{c,t} \cdot C_{c,t})
\]
其中 \( P_{g,t}, P_{c,t} \) 分别为绿色电力和传统电力使用量,\( C_{g,t}, C_{c,t} \) 分别为对应电价。

\end{CJK}

\end{document}